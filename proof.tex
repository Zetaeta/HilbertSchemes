\documentclass[a4paper,12pt,titlepage]{article}
\usepackage{amsmath}
\usepackage{amssymb}
\usepackage{amsfonts}
\usepackage{amsthm}
\usepackage{graphicx}
\usepackage{mathtools}
%\usepackage{newtxtext}
%\usepackage{newtxmath}
\usepackage{bm}
\usepackage[margin=0.75in]{geometry} \usepackage{polynom}

\newcommand{\tr}{\mbox{tr}}
\newcommand{\orr}{\mbox{ or }}
\newcommand{\adj}{\mbox{adj}}
\newcommand*\R[1]{$\mathbb{R}^{#1}$}
\newcommand{\C}{\mathbb{C}}
\newcommand{\p}{\mathfrak{p}}
\newcommand{\Z}{\mathbb{Z}}
\newcommand*\DD[2]{\frac{\partial#1}{\partial#2}}
\newcommand*\vc[1]{\mathbf{#1}}
\newcommand*\tens[1]{\mathop{\otimes}\limits_{#1}}
\newcommand*\vccrd[1]{\left(\begin{array}{c}#1\end{array}\right)}
\newcommand*\mat[2]{\left(\begin{array}{#1}#2\end{array}\right)}

\newcommand{\uvci}{{\bm{\hat{\textnormal{\bfseries\i}}}}}
\newcommand{\uvcj}{{\bm{\hat{\textnormal{\bfseries\j}}}}}
\DeclareRobustCommand{\uvc}[1]{{%
  \ifcat\relax\noexpand#1%
      % it should be a Greek letter
      \bm{\hat{#1}}%
      \else
      \ifcsname uvc#1\endcsname
        \csname uvc#1\endcsname
      \else
        \bm{\hat{\mathbf{#1}}}%
        \fi
  \fi
}}
\DeclareMathOperator{\Ker}{Ker}
\DeclareMathOperator{\Spec}{Spec}
\DeclareMathOperator{\Supp}{Supp}
\DeclareMathOperator{\Ann}{Ann}
\DeclareMathOperator{\im}{Im}
\DeclareMathOperator{\Hom}{Hom}

\newtheorem{thm}{Theorem}
\newtheorem{lmm}{Lemma}

\begin{document}

\begin{thm} Let $I \subset R := \C[x,y,z]$ is an ideal generated by monomials of the form $x^iz^j$ and $y^iz^j$
such that $R/I$ is a vector space of dimension $d$. Then $\dim_\C \Hom_R(I,R/I)=3d$
\end{thm}
$Hom(I,R/I)$ is generated by those morphisms that send the canonical generators of $I$ to monomials in $R/I$.
These morphisms can be represented as a collection of arrows in the 3 dimensional Young diagram made up of the set $S$ of monomials in $R/I$
with tails at the generators and heads at some block inside the diagram.
The map $f$ represented by such a collection of arrows is an $R-$morphism iff for any generators $\alpha,\beta$ such that $m\alpha=n\beta$, $mf(\alpha)=nf(\beta) \in R/I$.
This is satisfied when if some arrow of $f$, any arrows obtained by dragging that arrow block by block until its tail is at a different generator
while keeping its head inside $S$ and tail outside are also part of $f$, and there are no arrows that
can be dragged in this manner until their head lies below the bounds of the diagram (i.e. has negative powers of some variable).
We may assume that such a dragging happens along the boundary keeping the tail directly or diagonally adjacent to a block in $S$.

We may thus consider only morphisms where every arrow is equal in length and direction and any arrow can be dragged as described to any other.
Label the canonical generators in two groups: $\alpha_0=z^p$ downwards in the $xz$ plane to $\alpha_m=x^q$, $\beta_0=\alpha_0$ downwards in the $yz$ plane to $\beta_o=y^r$. 
The absence of generators containing nonzero powers of x and y means that if we take the z axis to be vertical, each horizontal layer of blocks in the diagram is rectangular.
There are three types of arrows possible: firstly those arrows from the generators $\alpha_1, \dots, \alpha_m$ with heads having higher or equal powers of $z$,
secondly arrows from the generators $\beta_1, \dots, \beta_o$ with the same condition,
and thirdly all arrows from any generators (except $\alpha_m$, $\beta_o$) with heads having smaller powers of $z$.

Consider the first class of arrows.
For each $\alpha_i$, $i=1, \dots, n$ let us count the morphisms $f$ consisting of these arrows such such that $f(\alpha_i) \neq 0$ and $f(\alpha_j)=0$ for $j<i$.
Let $p_i$ be the vertical distance from $\alpha_i$ to $\alpha_{i-1}$, and $q_i$ be such that $y^{q_i-1}\alpha_i/x \notin I$ but $y^{q_i}\alpha_i/x \in I$,
i.e. the distance in the $y$ direction from $\alpha_i$ along the edge of the diagram. $f(\alpha_{i-1})=0$ means that we cannot drag the arrow at $\alpha_i$ to $\alpha_{i-1}$,
therefore $z^{p_i}f(\alpha_i) \in I$. Such an arrow cannot be dragged to any generator $\beta_j$: 
$f(\alpha_i)$ has higher or equal powers of $z$ and $y$ than $\alpha_i$ (as $y \nmid \alpha_i$) so must have a lower power of $x$. Thus $f(\alpha_i)=y^bz^c\alpha_i/x^a$,
and dragging the arrow will produce arrows with tails at $m \in I$ and heads at $my^bz^c/x^a$. For any such arrow, $my^bz^c/x^a \in S \implies m/x^a \in S$.
The intersection of the horizontal layer containing $m$ with $S$ is rectangular and the tail of these arrows will thus always be found some power of $x$ beyond this rectangle
so cannot be dragged around to any $\beta_i$ on the $y$ side.

The head of the arrow has not smaller powers of $y$ and $z$ and $\alpha_j$ has higher powers of $x$ than $\alpha_i$ for $j>i$,
so dragging the arrow to these $\alpha_j$ cannot push the head below the diagram.
Therefore each arrow from $\alpha_i$ to some element of $S$ with a not lower power of $z$ that is at most $p_i$ blocks below the surface of the diagram
defines a distinct morphism that is zero on each $\alpha_j, j<i$ and on all $\beta_j$.
The number of these arrows is the same as the number of boxes in $S$ contained in the height interval from $\alpha_i$ to $\alpha_{i-1}$,
i.e. those whose powers of $z$ range from $a$ to $a+p_i-1$ if $a$ is the power of $z$ in $\alpha_i$.
Counting these boxes means that every box will be counted exactly once when we count $\alpha_1, \dots, \alpha_m$, giving $d$ morphisms of this type.

Morphisms corresponding to the second class of arrows are counted the same way interchanging $x$ and $y$, giving $d$ of them.

For each generator (except $\alpha_m, \beta_r$) we count morphisms of this type which take that generator to an element of $S$ and all generators with lower powers of $z$ to zero.
\begin{lmm} If an arrow of this type cannot be dragged to some lower generator directly (i.e. without dragging it upwards then down), then it cannot be dragged to a lower generator in any manner.
\end{lmm}


\end{document}
